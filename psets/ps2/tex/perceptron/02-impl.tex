\item \subquestionpoints{10} Implement your approach by completing the
\texttt{initial\_state}, \texttt{predict}, and \texttt{update\_state} methods
of \texttt{src/perceptron/perceptron.py}.


We provide three functions to be used as kernel, a dot-product kernel defined as:
\begin{align}
	K(x,z) = x^\top z,
\end{align}
a
radial basis function (RBF) kernel, defined as:
\begin{align}
K(x,z) = \exp \left (-\frac{\|x-z\|_2^2}{2\sigma^2}\right), 
\end{align}
and finally the following function:
\begin{align}
K(x,z) = \begin{cases}
	 -1 & x=z \\
	 0 & x\neq z
	\end{cases}
\end{align}
\sloppy Note that the last function is not a kernel function (since its corresponding matrix is not a PSD matrix).
However, we are still interested to see what happens when the kernel is invalid.
%
 Run \texttt{src/perceptron/perceptron.py} to train
kernelized perceptrons on \texttt{src/perceptron/train.csv}. The code will then test
the perceptron on \texttt{src/perceptron/test.csv} and save the resulting
predictions in the \texttt{src/perceptron/} folder. Plots will also be saved in
\texttt{src/perceptron/}.


Include the three plots (corresponding to each of the kernels) in your writeup,
and indicate which plot belongs to which function.
